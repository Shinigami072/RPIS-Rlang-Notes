% This file was converted to LaTeX by Writer2LaTeX ver. 1.4
% see http://writer2latex.sourceforge.net for more info
\documentclass[fleqn]{article}
\usepackage{amsmath}
\usepackage{amssymb,amsfonts,textcomp}
\usepackage[utf8]{inputenc}
\usepackage[T1]{fontenc}
\usepackage[english,polish]{babel}

\usepackage{array}



\title{Statystyka: Matematyka}
\author{Krzysztof Stasiowski}
\date{}
\begin{document}
\section*{Średnia:}
\[\bar{x} = \frac{1}{n}\sum_{i=1}^Nx_i\]
\[x_i \texttt{- wartość i}\]
\[n   \texttt{- ilość wartości }\]

\section*{Wariancja:}
\[s^2 = \frac{1}{n}\sum_{i=1}^N(x_i-\bar{x})^2\]
\[x_i \texttt{- wartość i}\]
\[n   \texttt{- ilość wartości }\]

\section*{Odchylenie Standardowe \(\sigma\):}
\[s = \sqrt{\frac{1}{n}\sum_{i=1}^N(x_i-\bar{x})^2}\]
\[x_i \texttt{- wartość i}\]
\[n   \texttt{- ilość wartości }\]

\pagebreak
\section*{Zmienna Dyskretna:}
Właściwości: suma prawdopodobieństw wszystkich zmiennych \(\sum_{i=1}{p_i}=1\).
\subsection*{Wartość Oczekiwana:}
\[E(X) = \sum_{i=1}^Nx_ip_i\]
\[p_i \texttt{- prawdopodobieństwo i wartość}\]
\[x_i \texttt{- wartość i}\]
\[n   \texttt{- ilość wartości }\]
\subsection*{Wariancja:}
\[V(X) = \sum_{i=1}^N(x_i-E(X))^2p_i = E(X^2) - [E(X)]^2\]
\[E(X) \texttt{- wartość spodziewana}\]
\[p_i \texttt{- prawdopodobieństwo i wartość}\]
\[x_i \texttt{- wartość i}\]
\[n   \texttt{- ilość wartości }\]
\subsection*{Wzór Bayesa:}
\[P(A|B) = \frac{P(A)P(B|A)}{P(B)}\]
\[P(A|B) \texttt{- prawdobodoieństwo A pod warunkem B}\]
\[P(B|A) \texttt{- prawdobodoieństwo B pod warunkem A}\]
\[P(A)   \texttt{- prawdobodoieństwo A}\]
\[P(B)   \texttt{- prawdobodoieństwo B}\]

\subsection*{Rozkład dwumianowy:}
\[P(X=k) = \binom{N}{k}*p^k*(1-p)^{n-k}\]
\[N \texttt{- ilość prób}\]
\[k \texttt{- ilość sukcesów}\]
\[p \texttt{- prawdopobobieństwo sukcesu}\]

\subsection*{Rozkład Poissona:}
\[P(X=k) = \frac{\lambda^ke^{-\lambda}}{k!}\]
\[k \texttt{- ilość sukcesów}\]
\[\lambda \texttt{- wartość spodziewana}\]
\[\lambda = E(x) = Np\]
\pagebreak
\section*{Zmienna Ciągła:}
Pole pod wykresem gęstości prawdopodobieństwa \(=1\)
\[\int_{-\infty}^{\infty}{P(x)dx}=1\]

\subsection*{Wartość Oczekiwana \(\mu\) :}
\[E(x) = \int_{-\infty}^{\infty}{xP(x)dx}\]
\[P(x) \texttt{- prawdopodobienstwo przyjęca wartości }x\]
\subsection*{Dystrybuanta:}
\begin{align*}
F(x_k)& = \int_{-\infty}^{x_k}{P(x)dx}\\
P(x) & & \texttt{- prawdopodobienstwo wartości }x\\
P(a < x \geq b)&= F(b)-F(a) &\texttt{- prawdopodobienstwo wartości }x\\
\end{align*}



\subsection*{Rozkład Normalny:}
\begin{align*}
N(\mu,\sigma)&&\texttt{- rozkład normalny}\\
N(0,1)& &\texttt{- standardowy rozkład normalny}\\
\mu& &\texttt{- wartość oczekiwana}\\
\sigma& &\texttt{- odcylenie standardowe}\\
P(a < x \geq b) = F(b)-F(a)& &\texttt{- prawdopodobienstwo wartości }x\\
P(\mu -\sigma < x \geq \mu+\sigma) \approx 68\%& &\texttt{- prawdopodobienstwo wartości }x  \\
P(\mu -2\sigma < x \geq \mu+2\sigma) \approx 95\%& &\texttt{- prawdopodobienstwo wartości }x \\
P(\mu -3\sigma < x \geq \mu+3\sigma) \approx 99\%& &\texttt{- prawdopodobienstwo wartości }x \\
\end{align*}

\subsubsection*{Standaryzacja Rozkładu normalnego}
\[X = N(\mu,\sigma) \texttt{- Dane o podanym rozkładznie normalnym}\]
\[Z = N(0,1) \texttt{- dane o normalnym rozkładzie standardowym}\]
\[Z = \frac{X-\mu}{\sigma} = N(0,1) \]
\end{document}
