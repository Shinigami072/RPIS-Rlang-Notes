% This file was converted to LaTeX by Writer2LaTeX ver. 1.4
% see http://writer2latex.sourceforge.net for more info
\documentclass[fleqn]{article}
\usepackage{amsmath,amssymb,amsfonts,textcomp}
\usepackage{array}
\usepackage[utf8]{inputenc}
\usepackage[T1]{fontenc}
\usepackage[english,polish]{babel}


\title{Statystyka: Matematyka}
\author{Krzysztof Stasiowski}
\date{}
\usepackage[pdftex,pagebackref=false,unicode=true,pdfusetitle,colorlinks=true,linkcolor=blue]{hyperref}

\begin{document}
\section{Wstęp:}
\subsection{Cechy X:}
\begin{itemize}
\item mierzalne (ilościowe) - właściwości można zmierzyć i wyrazić za pomocą odpowiednich jednostek
\item porządkowe (podtyp ilościowych) - badają natężenie badanej właściwości przedstawionej w sposób opisowy - np. oceny 
\item niemierzalne (jakościowe) - zwykle określane słownie nie mają jednoznacznych wartości liczbowych np. (płeć firma)
\end{itemize}


\subsection{Kwantyle Q:}
Dzieli zbiorowość na części od względem ilości cech. 
\begin{itemize}
\item
$Q_1$ - dzieli na dwie części - w taki sposób że $25\%$ ma wartości niższe a $75\%$ wyższe.
\item
$Q_2$ połowa cech ma wartości mniejsze a połowa większe - mediana - wartość środkowa.  
\item
$Q_3$ - $75\%$ mniejszych cech, $25\%$ większych.
\end{itemize}

\subsection{Średnia:}
Przybliżenie wartości oczekiwanej. Suma kwadratów odległości od średniej jest minimalna.
\[\bar{x} = \frac{1}{n}\sum_{i=1}^Nx_i\]
$x_i$ - wartość $i$\\
$n$ - ilość wartości

\subsection{Wariancja \texorpdfstring{\(s^2\)}{V(X)}:}
Średnia z wartości odchyleń od średniej arytmetycznej.
\[s^2 = \frac{1}{n}\sum_{i=1}^N(x_i-\bar{x})^2\]
$x_i$ - wartość $i$\\
$n$ - ilość wartości


\subsection{Odchylenie Standardowe \texorpdfstring{\(\sigma\)}{s}:}
Pierwiastek kwadratowy z wariancji - Stanowi miarę zróżnicowania, przeciętne zróżnicowanie wartości cechy od średniej arytmetycznej.

\[s = \sqrt{\frac{1}{n}\sum_{i=1}^N(x_i-\bar{x})^2}\]
$x_i$ - wartość $i$\\
$n$ - ilość wartości \\
Około $\frac{2}{3}$ wszystkich jednostek cechy znajduje się w przedziale: $(\bar{x}-s,\bar{x}+s)$\\
Około $99\%$ wszystkich wartości znajduje się w przedziale:  $(\bar{x}-3s,\bar{x}+3s)$

\subsection{Rozstęp:}
różnica między wartością maksymalną a minimalną cechy.\\
rozstęp między-kwantylowy (IQR). $Q_3-Q_1$
\pagebreak

\section{Szeregi:}
\subsection{Szereg Szczegółowy:}
uporządkowany ciąg wartości cechy statystycznej

\subsection{Szereg Rozdzielczy:}
uzyskuje się dzieląc dane statystyczne na pewne kategorie i podając liczebność danych w każdej kategorii - reprezentowane za pomocą histogramu.
często zakłada się że $k=\sqrt{n}$\\ $k$ - liczba przedziałów,\\ $n$ - liczebność 
\subsubsection{Wariancja:}
\[s^2 = \frac{1}{n}\sum_{i=1}^N(\bar{x_i}-\bar{x})^2n_i\]
$\bar{x_i}$ -średnia wartości w przedziale $i$\\
$n_i$ - ilość wartości w przedziale $i$

\pagebreak
\section{Prawdopodobieństwo:}
\[P(B) = \sum_i{P(A_i)P(B|A_i)}\]
$P(B|A_i)$- prawdopodobieństwo $B$ pod warunkiem $A_i$\\
$P(A_i)$ - prawdopodobieństwo $A_i$ - rozdzielne warunki\\
$P(B)$ - prawdopodobieństwo $B$

Warunek unormowania prawdopodobieństw:
\[\sum_{k=0}{P_k}=1\]

\subsection{Wzór Bayesa:}
\[P(A|B) = \frac{P(A)P(B|A)}{P(B)}\]
$P(A|B)$ - prawdopodobieństwo $A$ pod warunkiem $B$\\
$P(B|A)$- prawdopodobieństwo $B$ pod warunkiem $A$\\
$P(A)$ - prawdopodobieństwo $A$\\
$P(B)$ - prawdopodobieństwo $B$

\pagebreak

\section{Zmienna Dyskretna:}
Takie zmienne, które przyjmują skończony przeliczalny zbiór możliwych wartości.
Właściwości: suma prawdopodobieństw wszystkich zmiennych \(\sum_{i=1}{p_i}=1\).
\subsection{Dystrybuanta:}
Funkcja przyporządkowująca skumulowaną wartość prawdopodobieństwa mniejszego od danego $x$
\[F(x)=P(X \leq x)=\sum_{x_i:x_i \leq x}p(x_i)\]
\subsection{Wartość Oczekiwana:}
Najbardziej prawdopodobna wartość, przyjmowana przez $X$
\[E(X) = \sum_{i=1}^nx_ip_i\]
$p_i$ - prawdopodobieństwo $i$ wartość\\
$x_i$ - wartość $i$\\
$n$   - ilość wartości

Własność wartości oczekiwanej:
\[E(aX+b)=aE(X)+b\]
\[E(X-E(X))=0\]
\[E(X+Y)=E(X)+E(Y)\]
\[E(XY)=E(X)E(Y)\]

\subsection{Wariancja:}
wartość przeciętna kwadratu odchylenia zmiennej losowej od jej wartości oczekiwanej.
\[V(X) = \sum_{i=1}^N(x_i-E(X))^2p_i = E(X^2) - [E(X)]^2\]
$E(X)$ - wartość spodziewana\\
$p_i$ - prawdopodobieństwo $i$ wartość\\
$x_i$ - wartość $i$\\
$n$   - ilość wartości

Własności wariancji:
\[V(aX+b)=a^2V(X)\]
\[V(X+Y)=V(X)+V(Y)\]
\[V(X-Y)=V(X)+V(Y)\]

\subsection{Rozkład dwumianowy:}
Prawdopodobieństwo $k$ sukcesów przy $N$ próbach.
\[P(X=k) = \binom{N}{k}p^k(1-p)^{N-k}\]
\[V(X)= Np(1-p)\]
\[E(X)=Np\]
\[F(X)=P(X \leq x) = \sum_{k \leq x}{\binom{N}{k}p^k(1-p)^{N-k}}\]
$N$ - ilość prób\\
$k$ - ilość sukcesów\\
$p$ - prawdopodobieństwo sukcesu

\subsection{Rozkład Poissona:}
Prawdopodobieństwo $k$ sukcesów przy dużej ilości prób.
\[P(X=k) = \frac{\lambda^ke^{-\lambda}}{k!}\]
$k$ - ilość sukcesów\\
$\lambda$ - wartość spodziewana
\[\lambda = E(x) = Np\]

\pagebreak
\section{Zmienna Ciągła:}
Przyjmuje wartości rzeczywiste z określonego przedziału.
Pole pod wykresem gęstości prawdopodobieństwa \(=1\)
\[\int_{-\infty}^{\infty}{P(x)dx}=1\]

\subsection{Dystrybuanta:}
Prawdopodobieństwo przyjęcia wartości mniejszej od $x$ jest pole od $-\infty$ do niej.
\begin{align*}
F(x_k)& = \int_{-\infty}^{x_k}{P(x)dx}\\
P(x) & & \texttt{- prawdopodobienstwo wartości }x\\
P(a < x \geq b)&= F(b)-F(a) &\texttt{- prawdopodobienstwo wartości }x\\
\end{align*}

\subsection{Wartość Oczekiwana \texorpdfstring{\(\mu\)}{E(X)} :}
\[E(x) = \int_{-\infty}^{\infty}{xP(x)dx}\]
$P(x)$ - prawdopodobieństwo przyjęcia wartości$x$

\subsection{Wariancja:}
\[V(X)=\int_{-\infty}^{\infty}\left[X-E(X)\right]^2P(x)dx\]



\subsection{Rozkład Normalny:}
\[N(\mu,\sigma) =  \frac{1}{\sqrt{2\pi}\sigma}e^{-\frac{1}{2}} \left( \frac{x- \mu}{\sigma} \right)^2 \]
\begin{align*}
N(\mu,\sigma)&&\texttt{- rozkład normalny}\\
N(0,1)& &\texttt{- standardowy rozkład normalny}\\
\mu& &\texttt{- wartość oczekiwana}\\
\sigma& &\texttt{- odcylenie standardowe}\\
P(a < x \geq b) = F(b)-F(a)& &\texttt{- prawdopodobienstwo wartości }x\\
P(\mu -\sigma < x \geq \mu+\sigma) \approx 68\%& &\texttt{- prawdopodobienstwo wartości }x  \\
P(\mu -2\sigma < x \geq \mu+2\sigma) \approx 95\%& &\texttt{- prawdopodobienstwo wartości }x \\
P(\mu -3\sigma < x \geq \mu+3\sigma) \approx 99\%& &\texttt{- prawdopodobienstwo wartości }x \\
\end{align*}

\subsubsection{Standaryzacja Rozkładu normalnego}
\[X = N(\mu,\sigma) \texttt{- Dane o podanym rozkładznie normalnym}\]
\[Z = N(0,1) \texttt{- dane o normalnym rozkładzie standardowym}\]
\[Z = \frac{X-\mu}{\sigma} = N(0,1) \]
\end{document}
